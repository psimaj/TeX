\documentclass{article}

\usepackage{ragged2e}
\usepackage[OT4]{polski}
\usepackage[utf8]{inputenc}
\usepackage{amsfonts}
\usepackage[shortlabels]{enumitem}
\usepackage{amsmath}
\usepackage[left=2cm, right=2cm, top=1.5cm, bottom=1.5cm]{geometry}
\pagenumbering{gobble}
\makeatletter
\renewcommand{\maketitle}{\bgroup\setlength{\parindent}{0pt}
	\textbf{\@title}\egroup
	\\
}
\makeatother
\begin{document}
	\title{\large Zadanie: WFO \\ Wiele Formuł}\maketitle
	\centering \rule{18cm}{0.5pt} \justifying
	\textbf{Dostępna pamięć: 4MB} \\
	
	%MIEJSCE NA TREŚĆ, AKAPITY RODZIELONE \smallbreak$
	Dana jest liczba $S$ złożona z cyfr od $1$ do $9$. Możesz wstawić znak $+$ na dowolną liczbę (nawet i zerową) pozycji pomiędzy sąsiadującymi cyframi tej liczby. Między każdą parą sąsiadujących cyfr wstawić można co najwyżej jeden znak $+$. W ten sposób z tej liczby powstanie fomuła złożona z paru dodawań, którą można obliczyć.
	\smallbreak
	Znajdź sumę wszystkich formuł utworzonych w ten sposób.
	\begin{flushleft}
		\LARGE \textbf{Wejście}
	\end{flushleft}
	\smallbreak
	
	%INFORMACJE O WEJŚCIU$
	Na wejściu zadana jest liczba $S$. Składa się z cyfr od $1$ do $9$ i zachodzi $1 \leq |S| \leq 10$.
	\begin{flushleft}
		\LARGE \textbf{Wyjście}
	\end{flushleft}
	\smallbreak
	
	%INFORMACJE O WYJŚCIU$
	Na wyjściu należy wypisać jedną liczbę: sumę wszystkich formuł utworzonych z $S$ według opisanych zasad.
	\begin{flushleft}
		\LARGE \textbf{Przykład}
	\end{flushleft}
	\smallskip 
	Dla danych wejściowych: \\
	%DANE WEJŚCIOWE PRZYKŁADU$
	$125$
	\smallskip \\
	poprawnym wynikiem jest: \\
	%WYNIK PRZYKŁADU$
	$176$
	\smallskip \\
	\textbf{Objaśnienie do przykładu:} Wszystkie możliwe formuły utworzone z $125$ to $125$, $1+25$, $12+5$, $1+2+5$. Ich suma wynosi $125+26+17+8=176$.
\end{document}
