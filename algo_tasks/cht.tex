\documentclass{article}

\usepackage{ragged2e}
\usepackage[OT4]{polski}
\usepackage[utf8]{inputenc}
\usepackage{amsfonts}
\usepackage[shortlabels]{enumitem}
\usepackage{amsmath}
\usepackage[left=2cm, right=2cm, top=1.5cm, bottom=1.5cm]{geometry}
\pagenumbering{gobble}
\makeatletter
\renewcommand{\maketitle}{\bgroup\setlength{\parindent}{0pt}
	\textbf{\@title}\egroup
	\\
}
\makeatother
\begin{document}
	\title{\large Zadanie: CHT \\ Chciwy Taksówkarz}\maketitle
	\centering \rule{18cm}{0.5pt} \justifying
	\textbf{Dostępna pamięć: 128MB} \\
	
	%MIEJSCE NA TREŚĆ, AKAPITY RODZIELONE \smallbreak$
	Sieć drogowa miasta \textit{Goldshire} składa się ze skrzyżowań połączonych jednokierunkowymi drogami. Zakładamy, że jeżeli \textbf{a} i \textbf{b} są skrzyżowaniami, to może istnieć co najwyżej jedna bezpośrednia droga z \textbf{a} do \textbf{b}, ale może istnieć jednocześnie droga z \textbf{a} do \textbf{b} i z \textbf{b} do \textbf{a}, oraz nie istnieje bezpośrednia droga z \textbf{a} do \textbf{a}.
	\smallbreak
	Artur jest taksówkarzem w \textit{Goldshire}. Płacą mu od kilometra, więc nie idzie klientom na rękę i zawsze stara obrać się jak najdłuższą trasę do celu. Klienci są jednak sprytni i na każdym skrzyżowaniu na trasie sprawdzają pozostałą odległość do celu. Jeżeliby dostrzegli, że na kolejnych skrzyżowaniach trasy ta odległość nie maleje, to z pewnością donieśliby na Artura za oszustwo, a ten zostałby zwolniony, zatem nie może on dopuścić do takiej sytuacji.
	\smallbreak
	Dla zadanej sieci drogowej, wraz z punktami startowym i docelowym trasy Artura, znajdź długość wyżej opisanej najdłuższej trasy przejazdu.
	\begin{flushleft}
		\LARGE \textbf{Wejście}
	\end{flushleft}
	\smallbreak
	
	%INFORMACJE O WEJŚCIU$
	Pierwsza linia wejścia zawiera cztery liczby całkowite $n$, $m$, $s$, $t$, oznaczające kolejno liczbę skrzyżowań, liczbę dróg, numer skrzyżowania startowego i numer skrzyżowania docelowego. Zachodzi $1 < n \leq 10^6$, $1 \leq m \leq 10^6$, $1 \leq s \leq n$, $1 \leq t \leq n$, $s \neq t$. \\
	Kolejne $m$ linii wejścia opisuje drogi \textit{Goldshire}. Każda z nich zawiera trzy liczby całkowite $a$, $b$, $w$, oznaczające, że ze skrzyżowania $a$ wychodzi droga do skrzyżowania $b$ o długości $w$. Zachodzi $1 \leq a \leq n$, $1 \leq b \leq n$, $a \neq b$, $1 \leq w \leq 10^3$. \\
	Zagwarantowane jest, że istnieje przynajmniej jedna ścieżka z $s$ do $t$.
	\begin{flushleft}
		\LARGE \textbf{Wyjście}
	\end{flushleft}
	\smallbreak
	
	%INFORMACJE O WYJŚCIU$
	Na wyjście należy wypisać dokładnie jedną liczbę: długością najdłuższej trasy dla Artura.
	\begin{flushleft}
		\LARGE \textbf{Przykład}
	\end{flushleft}
	\smallskip 
	Dla danych wejściowych: \\
	%DANE WEJŚCIOWE PRZYKŁADU$
	$6$ $8$ $1$ $4$ \\
	$1$ $2$ $2$ \\
	$2$ $1$ $7$ \\
	$2$ $3$ $3$ \\
	$3$ $4$ $1$ \\
	$1$ $5$ $3$ \\
	$5$ $4$ $50$ \\
	$2$ $6$ $5$ \\
	$6$ $4$ $3$
	\smallskip \\
	poprawnym wynikiem jest: \\
	%WYNIK PRZYKŁADU$
	$10$
	\smallskip \\
	\textbf{Objaśnienie do przykładu:} Trasa wynikowa prowadzi kolejno przez skrzyżowania $1$, $2$, $6$, $4$. Trasa $1$, $5$, $4$ jest dłuższa, ale odległość $1$ od celu wynosi $6$, a odległość $5$ od celu wynosi $50$, zatem nie spełnia warunku malejących odległości.
\end{document}
