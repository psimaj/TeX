\documentclass{article}

\usepackage{polski}
\usepackage[utf8]{inputenc}
\usepackage{amsfonts}
\usepackage[shortlabels]{enumitem}
\usepackage{amsmath}
\usepackage[left=2cm, right=2cm, top=1.5cm, bottom=1.5cm]{geometry}
\pagenumbering{gobble}
\begin{document}{\begin{center}\LARGE Niezmienniki i kolorowania \\\large Przemysław Simajchel, warsztaty PWSZ 2018 \smallskip\end{center}}
		\begin{enumerate}[\Roman*.]
			\item Niezmienniki i (i półniezmienniki)
			
			\begin{enumerate}[1.]		
				%Iloczyn
				\item Na kartce zapisane jest $10$ liczb $1$ oraz $15$ liczb $-1$. W każdym kroku Jaś wybiera dwie spośród tych liczb, jedną zmazuje, a drugą zastępuje ich iloczynem. Powtarza tą czynność dopóki nie zostanie tylko jedna liczba. Jaka to liczba?
				
				%Rabka 2, parzystość
				\item Zadany jest ciąg liczb $1, 2, \ldots , 2018$. W każdym kroku wybieramy dowolne dwie liczby z tego ciągu, usuwamy je i dostawiamy do naszego ciągu wartość bezwzględną ich różnicy. Czy jest możliwe, aby po pewnej liczbie kroków otrzymać ciąg złożony z pojedynczego zera?
				
				%Rabka 2, reszta mod 3
				\item Mamy $4$ kawałki papieru. W każdym kroku wybieramy jakiś z kawałków i rozrywamy go na $4$ kolejne. Czy jest możliwe, aby po pewnej liczbie kroków posiadać dokładnie $2018$ kawałków papieru?

				%Neugebauer + aut, parzystość + geo
				\item (Aut na Neugebauerze) W punktach $\{(0, 0), (1, 0), (0, 1)\}$ układu współrzędnych siedzą trzy pchły. W każdym kroku pewna pchła $A$ przeskakuje nad inną pchłą $B$ w taki sposób, że $B$ jest środkiem odcinka wyznaczonego przez punkty wyskoku i lądowania $A$. Czy pchły mogą tak wybrać kolejne skoki, aby po ich wykonaniu znajdowały się w punktach $\{(0, 0), (2, 0), (0, 2)\}$ albo $\{(1, 1), (1, 0), (0, 1)\}$? 

				%Suma-różnica, Engel
				\item Koło podzielono na $6$ części. W pierwszej i trzeciej części wpisano liczbę $1$, w pozostałe $0$. W każdym kroku możemy powiększyć liczby w dwóch sąsiednich ćwiartkach o $1$. Czy po pewnej liczbie kroków możemy zrównać wszystkie liczby na kole?

				%PWSZ, Trikowa suma kwadratów	
				\item Wyobraźmy sobie, że na każdym punkcie na osi liczb naturalnych kładziemy pewną liczbę monet (być może zerową). W każdym ruchu możemy wybrać dowolny punkt $n > 0$ na osi, na którym znajdują się co najmniej $2$ monety, a następnie przełożyć jedną monetę z $n$ na $n-1$ i jedną na $n+1$. Załóżmy, że wykonaliśmy przynajmniej jeden ruch. Czy możemy wrócić do stanu początkowego po pewnej skończonej liczbie ruchów?

				%102, półniezmiennik, suma
				\item Dane jest $n$ liczb całkowitych nieujemnych $a_1, \ldots a_n$. W pierwszym kroku Jaś wybiera liczbę $a_i$, nazywa ją \textit{sporą}, a następnie wybiera dowolną inną liczbę $a_j$ i zmienia jej wartość na dowolną całkowitą nieujemną mniejszą od \textit{sporej} liczby. W każdym kolejnym kroku jak wybiera nową \textit{sporą} liczbę, ale tym razem zmienia wartość już nie dowolnie wybranej liczby, a takiej, która była \textit{spora} w poprzednim kroku. Czy Jaś może się bawić w ten sposób w nieskończoność?
					
				%Engel, reszta mod 4
				\item Każda z liczb $a_1, a_2, \ldots , a_n$ jest równa $1$ albo $-1$. Wiadomo, że $a_1 a_2 a_3 a_4 + a_2 a_3 a_4 a_5 + \ldots + a_n a_1 a_2 a_3 = 0$. Udowodnij, że $n$ jest podzielne przez $4$.
				
				%102, suma dwumianów
				\item W bloku o $120$ apartamentach mieszka $119$ osób. Powiemy, że apartament jest $przepełniony$, jeżeli mieszka w nim co najmniej $15$ osób. Każdego dnia mieszkańcy jednego z $przepełnionych$ apartamentów kłócą się i każdy przeprowadza się do innego mieszkania. Czy po pewnej liczbie dni żaden apartament nie będzie przepełniony?
				
				%Engel, suma kwadratów
				\item Dane są liczby $(3, 4, 12)$. W każdym kroku możemy wybrać dwie z nich, $a$ oraz $b$, i zastąpić je liczbami odpowiednio $\frac{3}{5}a - \frac{4}{5}b$, $\frac{4}{5}a + \frac{3}{5}b$. Czy po pewnej liczbie kroków możemy uzyskać liczby $(4, 6, 12)$?
					
				%Rabka 1, parzystość
				\item Na każdej godzinie na tarczy zegara wkręcono żarówkę. W każdym kroku możemy wybrać $6$ kolejnych żarówek i odwrócić stan (zgaszona/zapalona) każdej z nich. Na początku świeci się tylko żarówka na godzinie $12$. Czy jest możliwe, aby po pewnej liczbie kroków świeciła się tylko żarówka na godzinie $11$?
				
				%Rabka 2, parzystość
				\item Dana jest szachownica $8 \times 8$ ze skoczkiem w lewym górnym rogu. W każdym kroku wykonujemy nim ruch na pole, którego jeszcze nie odwiedziliśmy. Czy jest możliwe, aby zakończyć ten proces w prawym dolnym rogu szachownicy po odwiedzeniu wszystkich pól?
				
				%Rabka 2, półniezmiennik
				\item Dana jest szachownica $8 \times 8$, w której każdym polu znajduje się liczba rzeczywista. W każdym kroku możemy wybrać jeden wiersz lub kolumnę i odwrócić znaki wszystkich liczb, które się w niej znajdują. Pokaż, że za pomocą takich kroków można doprowadzić do sytuacji, kiedy suma wszystkich liczb na polach szachownicy będzie nieujemna.
				
				%Rabka 1, parzystość w kwadratach 2x2
				\item Dana jest szachownica $4 \times 4$, na której jedno pole w pierwszym wierszu nie będące narożnikiem jest czarne, a reszta pól biała. W każdym kroku możemy wybrać dowolny wiersz, kolumnę lub prostą równoległą do przekątnej i odwrócić kolory na wszystkich jej polach. Czy jest możliwe, by po pewnej liczbie kroków otrzymać całą białą planszę?
			\end{enumerate}
			\pagebreak
			\item Kolorowania (i niezmienniki)
			
			\begin{enumerate}[1.]
				%SEM, kolorowanie, parzystość
				\item Dzielimy okrąg na $10$ sektorów za pomocą $5$ różnych przekątnych. Początkowo w każdym sektorze kładziemy po monecie. W każdym kroku wybieramy dwa sektory. Najpierw z pierwszego przenosimy jedną monetę do kolejnego sektora zgodnie z kierunkiem wskazówek zegara, a następnie z drugiego przenosimy do kolejnego sektora przeciwnie do kierunku wskazówek zegara. Rozstrzygnij, czy po pewnej liczbie kroków można przenieść wszystkie monety do jednego sektora.
						
				%Engel, parzystość
				\item Czy planszę $8 \times 8$ z wyciętymi dwoma przeciwległymi rogami da się pokryć standardowymi dominami $1 \times 2$?
				
				%Engel, ciekawe kolorowanie
				\item Prostokątną podłogę pokryto pewną liczbą płytek $2 \times 2$ i $1 \times 4$. Niestety po jakimś czasie jedna z nich została rozbita, a jedyna płytka, jaką mamy w zapasie jest typu przeciwnego do rozbitej płytki. Czy jest możliwe, aby tak poprzestawiać płytki, by znów pokryć całą podłogę?

				%Paski, parzystość
				\item Czy planszę $10 \times 10$ da się pokryć $4$-blokowymi dominami w kształcie litery $L$?

				%Rabka 1, trójkolorowanie
				\item Na nieskończonej szachownicy wydzielony jest prostokąt $10 \times 9$, na którego każdym polu stoi pionek. Pionki nie mogą normalnie się poruszać, a jedynie zbijać poprzez przeskoczenie z aktualnego pola na pole za pionkiem sąsiadującym w pionie lub poziomie. Czy jest możliwe, aby po pewnym czasie na planszy pozostał tylko jeden pionek?
				
				%Rabka 1, rozciągnięte normalnie kolorowanie
				\item Czy można pokryć szachownicę $13 \times 13$ z wyciętym środkowym polem za pomocą bloków o wymiarach $1 \times 4$?
				
				%Engel, p-kolorowanie
				\item Niech $p$ będzie liczbą pierwszą. Pokaż, że planszę $a \times b$ da się pokryć prostokątami $1 \times p$ wtedy i tylko wtedy, gdy $p$ dzieli $a$ lub $p$ dzieli $b$.

			\end{enumerate}
		\end{enumerate}
\end{document}
