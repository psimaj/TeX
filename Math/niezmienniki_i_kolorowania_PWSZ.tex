\documentclass{article}

\usepackage{polski}
\usepackage[utf8]{inputenc}
\usepackage{amsfonts}
\usepackage[shortlabels]{enumitem}
\usepackage{amsmath}
\usepackage[left=2cm, right=2cm, top=1.5cm, bottom=1.5cm]{geometry}
\pagenumbering{gobble}
\begin{document}{\begin{center}\LARGE Niezmienniki i kolorowania \\\large Przemysław Simajchel, warsztaty PWSZ 2018 \smallskip\end{center}}
	\begin{enumerate}[\large \Roman*.]
		\item {\large Zadania} 
		
		\begin{enumerate}[\roman*.]
			\item Niezmienniki
			
			\begin{enumerate}[1.]
				\item (Engel) Koło podzielono na $6$ części. W pierwszej i trzeciej części wpisano liczbę $1$, w pozostałe $0$. W każdym kroku możemy powiększyć liczby w dwóch sąsiednich ćwiartkach o $1$. Czy po pewnej liczbie kroków możemy zrównać wszystkie liczby na kole?
				
				\item (Engel) Każda z liczb $a_1, a_2, \ldots , a_n$ jest równa $1$ albo $-1$. Wiadomo, że $a_1 a_2 a_3 a_4 + a_2 a_3 a_4 a_5 + \ldots + a_n a_1 a_2 a_3 = 0$. Udowodnij, że $n$ jest podzielne przez $4$.
				
				\item (102) Dane jest $n$ liczb całkowitych nieujemnych $a_1, \ldots a_n$. W pierwszym kroku Jaś wybiera liczbę $a_i$, nazywa ją $sporą$, a następnie wybiera dowolną inną liczbę $a_j$ i zmienia jej wartość na dowolną całkowitą nieujemną mniejszą od $sporej$ liczby. W każdym kolejnym kroku jak wybiera nową $sporą$ liczbę, ale tym razem zmienia wartość już nie dowolnie wybranej liczby, a takiej, która była $spora$ w poprzednim kroku. Czy Jaś może się bawić w ten sposób w nieskończoność?
				
				\item (102) W bloku o $120$ apartamentach mieszka $119$ osób. Powiemy, że apartament jest $przepełniony$, jeżeli mieszka w nim co najmniej $15$ osób. Każdego dnia mieszkańcy jednego z $przepełnionych$ apartamentów kłócą się i każdy przeprowadza się do innego mieszkania. Czy po pewnej liczbie dni żaden apartament nie będzie przepełniony?
				
				\item (Engel) Dane są liczby $(3, 4, 12)$. W każdym kroku możemy wybrać dwie z nich, $a$ oraz $b$, i zastąpić je liczbami odpowiednio $\frac{3}{5}a - \frac{4}{5}b$, $\frac{4}{5}a + \frac{3}{5}b$. Czy po pewnej liczbie kroków możemy uzyskać liczby $(4, 6, 12)$?
				
				\item Na tablicy zapisano liczbę $2018!$. W każdym kroku zmazujemy aktualnie zapisaną liczbę i zamiast niej zapisujemy sumę jej cyfr w systemie dziesiętnym. Powtarzamy tę procedurę dopóki na tablicy nie zostanie liczba jednocyfrowa. Jaka to liczba?
				
				\item Wyobraźmy sobie, że na każdym punkcie na osi liczb całkowitych kładziemy pewną liczbę monet (być może zerową). W każdym ruchu możemy wybrać dowolny punkt $n$ na osi, na którym znajdują się co najmniej $2$ monety, a następnie przełożyć jedną monetę z $n$ na $n-1$ i jedną na $n+1$. Załóżmy, że wykonaliśmy przynajmniej jeden ruch. Czy możemy wrócić do stanu początkowego po pewnej skończonej liczbie ruchów?
				
				\item Na kartce zapisane jest $10$ liczb $1$ oraz $15$ liczb $-1$. W każdym kroku Jaś wybiera dwie spośród tych liczb, jedną zmazuje, a drugą zastępuje ich iloczynem. Powtarza tą czynność dopóki nie zostanie tylko jedna liczba. Jaka to liczba?
				
				\item (Rabka 1) Na każdej godzinie na tarczy zegara wkręcono żarówkę. W każdym kroku możemy wybrać $6$ kolejnych żarówek i odwrócić stan (zgaszona/zapalona) każdej z nich. Na początku świeci się tylko żarówka na godzinie $12$. Czy jest możliwe, aby po pewnej liczbie kroków świeciła się tylko żarówka na godzinie $11$?
				
				\item (Rabka 2) Mamy $4$ kawałki papieru. W każdym kroku wybieramy jakiś z kawałków i rozrywamy go na $4$ kolejne. Czy jest możliwe, aby po pewnej liczbie kroków posiadać dokładnie $2018$ kawałków papieru?
				
				\item (Rabka 2) Dana jest szachownica $8 \times 8$ z koniem w lewym górnym rogu. W każdym kroku wykonujemy nim ruch na pole, którego jeszcze nie odwiedziliśmy. Czy jest możliwe, aby zakończyć ten proces w prawym dolnym rogu szachownicy odwiedziwszy po drodze wszystkie pola?
				
				\item (Rabka 2) Dana jest szachownica $8 \times 8$, w której każdym polu znajduje się liczba rzeczywista. W każdym kroku możemy wybrać jeden wiersz lub kolumnę i odwrócić znaki wszystkich liczb, które się w niej znajdują. Pokaż, że za pomocą takich kroków można doprowadzić do sytuacji, kiedy suma wszystkich liczb na polach szachownicy będzie nieujemna.
				
				\item (Rabka 2) Zadany jest ciąg liczb $1, 2, \ldots , 2018$. W każdym kroku wybieramy dowolne dwie liczby z tego ciągu, usuwamy je i dodajemy do naszego ciągu wartość bezwzględną z ich różnicy. Czy jest możliwe, aby po pewnej liczbie kroków otrzymać ciąg złożony z pojedynczego zera?
				
				\item (SEM) Dzielimy okrąg na $10$ obszarów za pomocą $5$ różnych przekątnych. Początkowo na każdym obszarze kładziemy po monecie. W każdym kroku wybieramy dwa sektory. Najpierw z pierwszego przenosimy jedną monetę do kolejnego sektora zgodnie z kierunkiem wskazówek zegara, a następnie z drugiego przenosimy do kolejnego sektora przeciwnie do kierunku wskazówek zegara. Rozstrzygnij, czy po pewnej liczbie kroków można przenieść wszystkie monety do jednego sektora.
				
				\item (SEM) Dany jest ciąg liczb $1, \frac{1}{2}, \frac{1}{3}, \ldots , \frac{1}{2009}$. W każdym kroku możemy z pozostałych w ciągu liczb wybrać dowolne dwie liczby $a$ i $b$, usunąć je i dodać do naszego ciągu liczbę $a + b + ab$. Po pewnym czasie zostanie tylko jedna liczba. Ile może ona wynosić?
				
				\item (Aut na Neugebauerze) W punktach $(0, 0)$, $(1, 0)$ i $(0, 1)$ układu współrzędnych siedzą trzy pchły. W każdym kroku pewna pchła $A$ przeskakuje nad inną pchłą $B$ w taki sposób, że $B$ jest środkiem odcinka wyznaczonego przez punkty wyskoku i lądowania $A$. Czy pchły mogą tak wybrać kolejne skoki, aby po ich wykonaniu znajdowały się w punktach $(0, 0)$, $(2, 0)$, i $(0, 2)$?
			\end{enumerate}
		
			\item Kolorowania (i niezmienniki)
			
			\begin{enumerate}[1.]
				\item Czy szachownicę $10 \times 10$ da się pokryć $4$-blokowymi dominami w kształcie litery $L$ w taki sposób, by żadne dwa na siebie nie zachodziły?
				
				\item (Rabka 1) Dana jest szachownica $4 \times 4$, na której jedno pole w pierwszym wierszu nie będące narożnikiem jest czarne, a reszta pól biała. W każdym kroku możemy wybrać dowolny wiersz, kolumnę lub prostą równoległą do przekątnej i odwrócić kolory na wszystkich jej polach. Czy jest możliwe, by po pewnej liczbie kroków otrzymać całą białą planszę?
				
				\item (Rabka 1) Na nieskończonej szachownicy wydzielony jest prostokąt $10 \times 9$, na którego każdym polu stoi pionek. Pionki nie mogą normalnie się poruszać, a jedynie zbijać poprzez przeskoczenie z aktualnego pola na pole za pionkiem sąsiadującym w pionie lub poziomie. Czy jest możliwe, aby po pewnym czasie na planszy pozostał tylko jeden pionek?
				
				\item (Rabka 1) Czy można pokryć szachownicę $13 \times 13$ z wyciętym środkowym polem za pomocą bloków o wymiarach $1 \times 4$ w taki sposób, aby żadne dwa bloki na siebie nie zachodziły?
			\end{enumerate}
		\end{enumerate}
	\end{enumerate}
\end{document}
